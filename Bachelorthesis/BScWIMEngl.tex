%%
%% 'BScWIMEngl.tex'
%%
%% LaTeX template for Bachelor theses at the faculty WIM
%% of Mannheim University
%%
%% copyleft (2012-2014) by J. Berger, T. Christen, and J. Potthoff	
%%

%%fleqn leftalignes equations
\documentclass[fleqn, 11pt, a4paper]{book}
\usepackage{./macros/dinaA4}
\usepackage{./macros/BScMacros}
\usepackage{./macros/BScTitle}

\usepackage[style=authoryear, backend=biber]{biblatex}
\addbibresource{./chapter/literature.bib}



%%
%% je nach Betriebssystem und Editor muss die
%% Behandlung von Sonderzeichen wie Umlauten
%% eingestellt werden - dazu unten das richtige 
%% (TeX-Guru fragen!) Paket auswaehlen 
%% (d.h., das "%" Zeichen am Anfang der Zeile entfernen)
%% (man geht auf "Nummer Sicher", indem man
%% Umlaute wie \"a eingibt, {\ss} fuer Eszett -
%% dann braucht man keins der Pakete unten)
%%
%\usepackage[cp850]{inputenc} 
%\usepackage[latin1]{inputenc} 
%\usepackage[ansinew]{inputenc}
\usepackage[utf8]{inputenc}
%\usepackage{ucs}
%\usepackage[utf8x]{inputenc}

\usepackage{csquotes}

%%
%% Font auswaehlen: '\TimesFont' auskommentiert -> computer modern
%% '\TimesFont' aktiv -> Textfont: times, Mathfont: computer modern
%%
%\TimesFont
%%
%% ein- oder doppelseitiger Satz
%%
%\onesided
\twosided
%%
%% deaktiviere die farbigen Rahmen um interne links (s. hyperref package):
%%
%%
%% persoenliche Angaben
%%
\firstLastName{Felix Benning}
\birthdate{27.11.1996}
\birthplace{Nürtingen}
\MatNr{1501817}
\titellineI{Markov-Decision Processes}
%\titellineII{}
%\titellineIII{}
%\subtitellineI{}
%\subtitellineII{}
\Supervisor{Prof. Dr. Leif Döring}
\DueDate{}


\begin{document}
\frontmatter
\titelpage
%%
%% eidestattliche Erklaerung
%%
\chapter*{Declaration of Authorship}
\thispagestyle{empty}

I hereby declare that the thesis submitted is my own unaided work. All direct or indirect sources used are acknowledged as references.

\vspace{.5\baselineskip}
This thesis was not previously presented to another examination board and has not been published.

\vspace{4\baselineskip}
\begin{center}
\parbox{.8\textwidth}{City, Date \hfill Signature}
\end{center}

\endinput

\cleardoublepage
%%
%% Vorwort
%%
%\chapter{Preface}

\endinput

\setcounter{tocdepth}{1}
%%
%% Inhaltsverzeichnis
%%
\tableofcontents
%%
%% Einleitung
%% (nicht alle Autoren trennen Vorwort und Einleitung,
%%  manche setzen die Einleitung als erstes Kapitel
%%  nach "\mainmatter")
%%
%% !TEX root = ../BScWIMEngl.tex

\chapter{Introduction}
    Let us consider a system to be a intelligent, if it collects information from its sensors and transforms this information into an ``appropriate'' response. Then the question where this input comes from (eyes, ears, cameras, etc.), or what kind of input it is (visual, audio, etc.), is as irrelevant for this definition, as the question what a response is (motor activity, sound, etc.). An intelligent system is simply a function mapping inputs to ``appropriate'' outputs.  The question what ``appropriate'' means, is in contrast a lot more important and difficult to answer. But if we assume that there exists a ``correct'' response, then this correct response is itself a function from the input space to the output space. Artificial intelligence is therefore simply an artificial implementation of this correct response function. 

    If we have complete knowledge of this function, we can encode this function as an executable program manually. But more often than not, we do not really understand the rules according to which the inputs are supposed to be transformed into outputs. 

    For this reason we might want a machine to ``learn'' these rules by itself, i.e. we want it to approximate the correct response function. There are two larger subcategories in this category of \emph{machine learning}. 
    
    In \emph{supervised learning} we do not know the correct response function, but we know the correct response for certain inputs intuitively. An example for this case is image classification of animals. There, we do not know how the input (pixels of the picture) map to the output (name of the animal), but we can tell for a given picture what animal is depicted intuitively, i.e. we can provide examples of the correct response function to the learning algorithm. Supervised Learning is therefore concerned with generalization from examples. And while approximating a function with a finite sample of (error prone) function evaluations is a relatively old problem in numerical analysis and statistics, supervised learning is often associated with more recent approaches like artificial neuronal networks. 

    While we are still able to provide examples in supervised learning, \emph{unsupervised learning} has to work with even less. This category includes clustering and principle component analysis, as well as \emph{reinforcement learning}. Reinforcement learning is applied to problems, where we ``know the correct response if we see it'', but can not provide examples ourselves. This is for example the case with walking. We know how walking looks like, but we have a hard time giving an example for the correct output signals sent to the motors in the leg. It is also used in cases where we do not know the correct response and have never seen it, but are able to compare different response functions. Examples range from games like chess, to profit maximization of a company. 

    In general reinforcement learning is used in cases, where humans or the environment in general can rate the solution, and provide rewards for good outcomes. Similar to the conditioning of animals, desirable actions are \emph{reinforced} with rewards. 

    But in order to write algorithms, which maximize their rewards, we first need to formulate the relationship of possibly delayed rewards with actions (outputs) of the learning algorithm in certain states (given certain inputs). The model for this relationship -- which virtually all modern reinforcement learning algorithms are based on -- is the Markov Decision Process. 
    
    We will therefore introduce this model and its properties in the first chapter, continue with a review of common reinforcement algorithms in the second chapter, and explore the relation between the theory of stochastic approximation and reinforcement learning in the third chapter, which yields proofs of convergence to the optimal policy for a number of reinforcement algorithms. \fxnote{check}
\endinput

\mainmatter
%%
%% eigentliche Kapitel der Arbeit
%%
\chapter{Markov Decision Processes}
\begin{definition}(Kernel)
	\((Y,\cA_Y), (X,\cA_X)\) measure spaces\\
	 \(\lambda\colon X\times\cA_Y\to \R\) is a \emph{(probability) kernel} \(
	 :\iff \begin{aligned}[t]
	 &\lambda(\cdot,A)\colon x\mapsto \lambda(x,A) \text{ measurable}\\
	 &\lambda(x,\cdot)\colon A\mapsto \lambda(x,A) \text{ a (prob.) measure}
	 \end{aligned}
	  \)\\
	  Since we will interpret probability kernels as distributions over \(Y\) given a  certain condition \(X\), the notation \(\lambda(\cdot\mid x) \coloneqq \lambda(x,\cdot)\) helps this intuition. 
\end{definition}
\begin{definition}(Markov Decision Process - MDP)\\
\(\cM=(\cX,\cA,\cP_0) \), with:
	
\begin{tabular}{l l l}
	\(\cX\) & countable (finite) set of states&\\
	\(\cA\) & countable (finite) set of actions &\\
	\multicolumn{2}{l}{
		\(
		\begin{cases}
		\cX\times \cA \to \pmeas(\cX\times\R)\\
		(x,a)\mapsto \cP_0(\cdot \mid x,a)
		\end{cases}
		\)
	}  &
	\parbox[h]{17em}{
		\emph{transition probability kernel} \\
		\(\pmeas(\cX\times\R) \) the set of probability measures on \(\cX\times\R \), \\
		\(\cX\) represents the next states,\\
		\(\R\) the payoffs
	}
	\end{tabular}\\
is a \emph{(finite) Markov Decision Process}.\\
Together with a discount factor \(\gamma\in[0,1]\) it is a:\\
\begin{tabular}{l l}
	\emph{discounted reward} MDP & \(\gamma <1 \)\\
	\emph{undiscounted reward} MDP & \(\gamma=1 \)
\end{tabular}\\
For \((Y_{(x,a)}, R_{(x,a)})\sim \cP_0(\cdot\mid x,a) \) a random variable, is
\[r(x,a)\coloneqq \E[R_{(x,a)}] \quad\text{ the \emph{immediate reward function}}\]
An MDP is \emph{evaluated} as follows:\\
1. Select the initial state \(X_0\) an \(\cX\)-valued random variable.\\ 
2. \((A_t, t\in \N)\) action selection rules (behaviors) will be discussed later, for now simply assume \(\cA\)-valued random variables.\\
3. Select inductively: \((X_{t+1}, R_{t+1})\sim\cP_0(\cdot\mid X_t, A_t)\) with the markov property, i.e.:
\begin{align*} 
&\Pr[(X_{t+1},R_{t+1})=(x,r) \mid (X_t,A_t)=(x_t,a_t),\dots, (X_0,A_0)=(x_0,a_0)]\\
&=\Pr[(X_{t+1}, R_{t+1})=(x,r)\mid (X_t,A_t)=(x_t,a_t)]
\end{align*}
resulting in the stochastic process \(((X_t,A_t,R_{t+1}), t\geq 0)\), which allows to define the \emph{return}:
\[\cR\coloneqq\sum_{t=0}^{\infty}\gamma^t R_{t+1}\]
\end{definition}
%(

\begin{remark}
\((X_{t+1}, R_{t+1})\sim\cP_0(\cdot\mid X_t, A_t)\) with the markov property is well defined, i.e.:
\begin{align*}
	&\exists (X_{t+1}, R_{t+1})\ \cX\times\R\text{-valued random variable}: \\ 
	&(X_{t+1}, R_{t+1})\sim\cP_0(\cdot\mid X_t, A_t) \text{ and satisfies the markov property}
\end{align*}
\end{remark}
\begin{proof}
	%TODO
\end{proof}
\begin{remark}\leavevmode
\begin{enumerate}
	\item From now on we assume that \(\forall (x,a)\in\cX\times\cA:|R_{(x,a)}|\le R\in\R\) almost surely. This also implies: 			\(
	\begin{aligned}[t]
		&\|r\|_{\infty}=\sup_{(x,a)\in\cX\times\cA}|\E[R_{(x,a)}]|\le R\\
		&|\cR|\le\sum_{t=0}^\infty \gamma^t |R_{t+1}|\le \frac{R}{1-\gamma} \text{ a.s.}
	\end{aligned}
	\)
	\item Sometimes not all actions make sense in all states. A simple fix would be to set the immediate reward functions for those actions very low, or (if possible) redirect them to the closest possible action. \\
	A more formal approach would be to introduce an additional mapping, which assigns the set of admissible actions to each state \(\cX\to\cP(\cA)\), or alternatively define a (binary) relation on \(\cX\times\cA\).
	\item If there is just one admissible action in every state, the MDP is equivalent to a normal Markov Process.
	\item Instead of a transition probability kernel \(\cP_0\), sometimes a \emph{transition function} f with a and an exogenous random element \(D_t\) (e.g. Demand) is used to define the next state and reward: \((X_{t+1},R_{t+1})=f(X_t,A_t,D_t)\) 
\end{enumerate}
\end{remark}
\begin{definition}\(\cM=(\cX,\cA,\cP_0)\) a MDP\\
	\(x\in\cX\) is a \emph{terminal (absorbing)} state \(:\iff \forall s\in\N: \Pr(X_{t+s}=x\mid X_t=x)=1\)\\
	An MDP with such states is called \emph{episodic}.\\
	An \emph{episode} is the random time period \((1,\dots,T)\) until a terminal state is reached.
\end{definition}
\begin{remark}\leavevmode
	\begin{itemize}
		\item The reward in a terminal state is by convention zero, i.e. \(x\) terminal state implies \(\forall a\in\cA: R_{(x,a)}=0\).
		\item Episodic MDPs are often undiscounted
	\end{itemize}	
\end{remark}
\begin{definition}\(\cM=(\cX,\cA,\cP_0)\) a MDP\\
	 An \(A_t\) selection-rule \(\pi=(\pi_t,t\in\N_0)\) is called \emph{behavior}, where
	 \[ 
	 	\pi_t\colon
	 	\begin{cases}
	 		((\cX\times\cA\times\R)^t\times\cX)\times\cP(\cA) \to \R \\
	 		(y,A)\mapsto \pi_t(A\mid y)
	 	\end{cases} \text{ is a probability kernel}
	 \]
	 and \(A_t\sim \pi_t(\cdot\mid (X_0,A_0,R_1), \dots,(X_{t-1},A_{t-1},R_t),X_t))\)\\
	 Special cases:
	 \begin{enumerate}
	 	\item \emph{Determinisitic stationary policies} specified with some abuse of notation:
	 	\[\leavevmode \pi\colon\cX\to\cA \text{ with }A_t=\pi(X_t)\]
	 	\item \emph{(Stochastic) stationary policies} specified by:
	 	\[\pi\colon \begin{cases}
	 	\cX\times\cP(\cA)\to\R\\
	 	(x,A)\mapsto \pi(A\mid x)
	 	\end{cases} \text{ with } A_t\sim\pi(\cdot\mid x)
	 	\]
	 \end{enumerate}
	 \(\statPolicy\) is the \emph{set of (stoch.) stationary policies}, \\ 
	 \(\detPolicy\) is the \emph{set of deterministic stationary policies} (note \(\detPolicy\subseteq\statPolicy \))
\end{definition}
\begin{remark}
A stationary policy induces a \emph{time-homogenous} markov chain.
\end{remark}
\begin{definition}(Markov Reward Process - MRP)
%TODO
\end{definition}
\section{Value functions}
The goal in this section is to
\begin{itemize}[itemsep=0pt, topsep=1pt]
\item define Value functions which assign states (and actions) a value, which allow the agent to make a more nuanced decisions than comparing immediate rewards of different actions
\item explore the relation of different value functions
\item show uniqueness of optimal value functions with the Banach fixpoint theorem, yielding a simple approximation methode along the way
\item demonstrate that in MDPs deterministic stationary policies are generally a large enough set of policies to choose from
 \end{itemize}
\begin{definition}\(\cM=(\cX,\cA,\cP_0)\) MDP, \(\pi\) Behavior\\
Select \(X_0\) such that \(\forall x\in\cX:\Pr(X_0=x)>0\) and evaluate the MDP with
\(((X_t,A_t,R_{t+1}), t\in \N_0)\) the resulting stoch. process.\\
\def\arraystretch{3}
\[
\begin{array}{l l}
	V^\pi\colon
	\begin{cases}
		\cX\to\R\\
		x\mapsto \E[\cR\mid X_0=x]
	\end{cases} 
	& \text{is the \emph{value function} for } \pi \footnotemark\\
	Q^\pi\colon
	\begin{cases}
		\cX\times\cA\to\R\\
		(x,a)\mapsto \E[\cR\mid X_0=x, A_0=a]
	\end{cases}
	& \text{is the \emph{action value function} for } \pi \footnotemark\\
	V^*\colon
	\begin{cases}
		\cX\to\R\\
		x\mapsto \sup\limits_{\pi\text{ Behav.}} V^\pi(x)
	\end{cases} 
	& \text{is the \emph{optimal value function}}\\
	Q^*\colon
	\begin{cases}
		\cX\times\cA\to\R\\
		(x,a)\mapsto \sup\limits_{\pi\text{ Behav.}} Q^\pi(x,a)
	\end{cases}
	& \text{is the \emph{optimal action value function}}
\end{array}
\]
\(\pi\) is \emph{optimal} \(:\iff V^*=V^\pi\)
\end{definition}
\footnotetext[1]{Well defined because \(\Pr(X_0=x)>0\)}
\footnotetext{Well defined because \(A_1\sim \pi_1(\cdot\mid (x,a,r_0), x_1)\) is defined for all \(a\) regardless of \(\pi_0\)}

\begin{remark}
With the distribution of \(X_0\) set (or \(X_0\) being realized with a fixed value \(x\)), the distribution of \(X_t, A_t,R_{t+1}\) is determined for all \(t\in\N_0\). The conditional expectation is thus unique for a given \(X_0=x\), for all possible realizations of the MDP with a given behavior. \\
This means \(V^\pi, Q^\pi\) are well defined.
\end{remark}


\begin{definition}\(\cM=(\cX,\cA,\cP_0)\) MDP\\
Sometimes we don't care about the probability distribution of the reward, so we define:
	\[
	p\colon 
	\begin{cases}
		\cX\times\cA\times\cP(\cX)\to\R\\
		(x,a,Y)\mapsto \cP_0(Y\times\R\mid x,a)
	\end{cases}\text{ the \emph{\underline{state} transition kernel}.}
	\] 
And use the notation \(p(y\mid x,a)\coloneqq p(\{y\}\mid x,a)\) with \((x,a,y)\in\cX\times\cA\times\cX\)
\end{definition}


\begin{prop}\label{expand Q^pi}\(\cM=(\cX,\cA,\cP_0)\) MDP, \(\pi\in\detPolicy\) 
	\[Q^\pi(x,a)=r(x,a)+\gamma\sum_{y\in\cX}p(y\mid x,a)V^\pi(y)	\]
\end{prop}

\begin{proof}
\begin{align*}
Q^\pi &= \E[\cR(\pi)\mid X_0=x, A_0=a]\\
&=\E[R_1(\pi)\mid X_0=x,A_0=a]+\gamma\E\left[\sum_{t=0}^\infty\gamma^t R_{t+2}(\pi)\middle| X_0=x,A_0=a\right]\\
&=\E[R_{(x,a)}] 
 + \gamma \sum_{y\in\cX}\E\left[\sum_{t=0}^\infty\gamma^t R_{t+2}(\pi) \middle| X_0=x,A_0=a, X_1=y\right]p(y\mid x, a)\\
&\lxeq{\text{Markov}} r(x,a)
 + \gamma\sum_{y\in\cX}\underbracket{\E\left[\sum_{t=0}^\infty\gamma^t R_{t+2}(\pi)\middle| X_1=y, A_1=\pi(y)\right]}_{
 \begin{aligned}
 	&= \E\left[\sum_{t=0}^\infty\gamma^t R_{t+2}(\pi)\middle| X_1=y \right] \\
 	&\lxeq{(*)} \E\left[\sum_{t=0}^\infty\gamma^t \tilde{R}_{t+1}(\pi)\middle| \tilde{X}_0=y \right]=V^\pi(y)
 \end{aligned}
 }
 p(y\mid x, a)
\end{align*}
\((*)\) Rename: \(\tilde{X}_{t}\coloneqq X_{t+1}, \tilde{A}_t\coloneqq A_{t+1},\tilde{R}_{t}\coloneqq R_{t+1}\), then \((\tilde{X}_t,\tilde{A}_t,\tilde{R}_{t+1}, t\in\N_0)\) is an evaluation of the MDP with the (stationary) policy \(\pi\)
\end{proof}


\begin{corollary}\label{V^pi,Q^pi relation}  \(\cM=(\cX,\cA,\cP_0)\) MDP, \(\pi\in\detPolicy\) 
\begin{align*}
	V^\pi(x)&=Q^\pi(x,\pi(x))\\
	 &=r(x,\pi(x))+\gamma\sum\limits_{y\in\cX}p(y\mid x,\pi(x))V^\pi(y) 
\end{align*}
\end{corollary}

\begin{proof}
Since \(\pi\) is a deterministic stationary policy:
\[V^\pi(x)=\E[\cR(\pi)\mid X_0=x]=\E[\cR(\pi)\mid X_0=x, A_0=\pi(x)]=Q^\pi(x,\pi(x))\]
The rest follows from \ref{expand Q^pi}
\end{proof}

\begin{definition}\(\cM=(\cX,\cA,\cP_0)\) MDP, \(\pi\in\detPolicy\) \\
The mapping \(T^\pi\colon\R^\cX\to\R^\cX\) with:
	\[
	T^\pi V(x)\coloneqq r(x,\pi(x))+\gamma\sum_{y\in\cX}p(y\mid x,\pi(x)) V(y)\qquad V\in\R^\cX, x\in\cX
	\]
is called the \emph{Bellman Operator}
\end{definition}


\begin{remark}\leavevmode
	\begin{enumerate}[label=\arabic*.]
	\item \(\forall\pi\in\detPolicy : T^\pi V^\pi=V^\pi\) (c.f. \ref{V^pi,Q^pi relation})
	\item\label{num:2} \(T^\pi\) meets the requirements of the Banach fixed-point theorem for \({\gamma<1}\), this implies that \(V^\pi\) for \(\pi\in\detPolicy\)
	is a \emph{unique} fixpoint and can be approximated with the canonical iteration
	\item \(T^\pi\) is an affine operator
	\item\label{num:4} \(W_1,W_2\in\R^\cX\), write \(W_1 \le W_2\) for \(\forall x\in\cX:W_1(x)\le W_2(x)\), then:
	\[W_1\le W_2 \implies T^\pi W_1\le T^\pi W_2\]
	\end{enumerate}
\end{remark}

\begin{proof}
\ref{num:2} \((\R^\cX, \|\cdot\|_\infty)\) is a non-empty, complete metric space and the mapping maps onto itself. It is left to show, that \(T^\pi\) is a contraction. Be \(V,W\in\R^\cX\):
\begin{align*}
	\|T^\pi V -T^\pi W\|_\infty &=\|\gamma\sum_{y\in\cX} p(y\mid \cdot,\pi(\cdot)) (V(y)-W(y))\|_\infty\\
	&\le\gamma \sup_{x\in\cX}\left\{ \sum_{y\in\cX} p(y\mid x,\pi(x)) \|V-W\|_\infty \right\}\\
	&=\gamma \|V-W\|_\infty  \sup_{x\in\cX}\underbracket{\left\{ \sum_{y\in\cX} p(y\mid x,\pi(x)) \right\}}_{=1}\\
	&=\gamma \|V-W\|_\infty 
\end{align*}
\ref{num:4} Be \(W_1,W_2\in\R^\cX\), \(W_1\le W_2\) and \(x\in\cX\):
\begin{align*}
	T^\pi W_2 (x) - T^\pi W_1(x) 
	= \gamma \sum_{y\in\cX} p(y\mid x,\pi(x)) \underbracket{(W_2(y)-W_1(y))}_{\ge 0} 
	\ge 0
\end{align*}
\end{proof}


\begin{definition}\(\cM=(\cX,\cA,\cP_0)\) MDP
\begin{align*}
	\tilde{V}(x)&\coloneqq \sup_{\pi\in\detPolicy} V^\pi(x)\\
	\tilde{Q}(x,a)&\coloneqq\sup_{\pi\in\detPolicy}Q^\pi(x,a)
\end{align*}
\end{definition}


\begin{definition}\(\cM=(\cX,\cA,\cP_0)\) MDP\\
The mapping \(T^*\colon\R^\cX\to\R^\cX\) with:
	\[
	 T^* V(x)\coloneqq \sup_{a\in\cA}\left\{r(x,a)+\sum_{y\in\cX}p(y\mid x,a) V(y)\right\} \qquad V\in\R^\cX, x\in\cX
	\]
is called the \emph{Bellman Optimality Operator}
\end{definition}


\begin{lemma}\label{V*,Q* relation}\(\cM=(\cX,\cA,\cP_0)\) MDP
\begin{enumerate}[label=\textbf{(\roman*)},font=\normalfont]
\item\label{i:1} \(\tilde{V}(x)=\sup\limits_{a\in\cA} \tilde{Q}(x,a)\)
\item\label{i:2} \( \tilde{Q}(x,a) = r(x,a)+\gamma\sum\limits_{y\in\cX}p(y\mid x,a)\tilde{V}(y) \)
\item\label{i:3} \(V^*(x)=\sup\limits_{a\in\cA}Q^*(x,a)\)
\item\label{i:4} \(Q^*(x,a)=r(x,a)+\gamma \sum\limits_{y\in\cX}p(y\mid x,a) V^*(y)\)
\end{enumerate}
\end{lemma}

\begin{proof} \ref{i:1}, \ref{i:2}
By \ref{V^pi,Q^pi relation} we know \(V^\pi(x)=Q^\pi(x,\pi(x))\) thus:
\begin{align*}
	\tilde{V}(x) =\sup_{\pi \in \detPolicy} V^\pi(x) \le \sup_{a\in \cA}\sup_{\pi\in\detPolicy} Q^\pi(x,a)
	=\sup_{a\in \cA} \tilde{Q}(x,a)
\end{align*}
Because of \ref{expand Q^pi} we know:
\begin{align*}
	\tilde{Q}(x,a)&=\sup_{\pi\in\detPolicy}Q^\pi(x,a) \\
	&= \sup_{\pi\in\detPolicy} \left\{ r(x,a)+\gamma\sum\limits_{y\in\cX}p(y\mid x,a) V^\pi(y) \right\}\\
	&\le r(x,a)+\gamma\sum\limits_{y\in\cX}p(y\mid x,a) \underbracket{\sup_{\pi\in\detPolicy} V^\pi(y)}_{\tilde{V}(y)} \\
\end{align*}
%TODO
\end{proof}


\begin{corollary}
	\begin{align*}
	&T^*\tilde{V}=\tilde{V}\\
	&T^*V^*=V^*
	\end{align*}
\end{corollary}

\begin{proof}
\begin{align*}
	V^*(x)\xeq{\ref{i:3}}\sup_{a\in\cA}Q^*(x,a)\xeq{\ref{i:4}}\sup_{a\in\cA}\left\{r(x,a)+\sum_{y\in\cX}p(y\mid x,a) V^*(y)\right\} =T^*V^*(x)
\end{align*}
\(\tilde{V}\) analogous
\end{proof}


\begin{thm}\(\cM=(\cX,\cA,\cP_0)\) MDP\\
\(T^*\) satisfies the requirements of the Banach fixpoint theorem, in particular:
	\[V^*(x)=\sup_{\pi\in\statPolicy}V^\pi(x)=\tilde{V}(x) \]
is the unique fixpoint of \(T^*\)
\end{thm}

\begin{lemma}(Blackwell's condition for contraction)
\end{lemma}

\begin{proof}
https://math.stackexchange.com/questions/1087885/blackwells-condition-for-a-contraction-why-is-boundedness-neccessary?rq=1
\end{proof}

\begin{proof}[Proof (Theorem)]
\end{proof}


\begin{prop}\label{sup is attained}\(\cM=(\cX,\cA,\cP_0)\) MDP\\
The following statements are equivalent:
\begin{enumerate}[label={(\roman*)},font=\normalfont]
\item \(\pi \in\statPolicy\) is optimal (\(V^*=V^\pi\))
\item \(\forall x\in\cX: V^*(x)=\sum\limits_{a\in\cA}\pi(a\mid x)Q^*(x,a)\)
\item\label{ii:3} \(\forall x\in\cX: \pi=\arg\max\limits_{\pi\in\statPolicy}\sum\limits_{a\in\cA}\pi(a\mid x)Q^*(x,a) \)
\item \(\pi(a\mid x)>0 \iff Q^*(x,a)=V^*(x) =\sup\limits_{b\in\cA}Q*(x,b)\) \\
	``actions are concentrated on the set of actions that maximize \(Q^*(x,\cdot)\)''\\
	(this also implies: \(Q^*(x,a)<V^*(x) \implies \pi(a\mid x)=0\))
\end{enumerate}
\end{prop}

\begin{proof}
%TODO
\end{proof}


\begin{definition}
	\(Q\colon\cX\times\cA\to\R\) an action value function, \(\tilde{\pi}\colon\cX\to\cA\) with:
	\[
	\tilde{\pi}(x)\coloneqq\arg\max_{\pi\in\statPolicy}\sum_{a\in\cA}\pi(a\mid x) Q(x,a)\qquad x\in\cX
	\]
	\(\tilde{\pi}(x)\) is called \emph{greedy} with respect to Q in \(x\in\cX\)\\
	\(\tilde{\pi}\) is called \emph{greedy} w.r.t. Q
\end{definition}


\begin{remark}\leavevmode
	\begin{itemize}
	\item \ref{sup is attained}\ref{ii:3} implies that greedy w.r.t. \(Q^*\) is optimal. 
	This means that knowledge of \(Q^*\) is sufficient to select the best action.
	\item \ref{V*,Q* relation} implies that knowledge of \(V^*,r,p\) is sufficient as well.
	\end{itemize}
\end{remark}
\endinput

% !TEX root = ../BScWIMEngl.tex  

\chapter{Reinforcement Learning Algorithms}

\section{Introduction}

Dynamic Programming usually breaks down in the real world for two reasons:
\begin{enumerate}
    \item The transition probabilities and immediate rewards are not known or hard to calculate.
    \item The state and action space is too large to even compute one iteration of Dynamic Programming for every state-action tuple (e.g. possible positions and possible moves in every position in chess).
\end{enumerate}

This is where algorithms which try to find good solutions without having to sweep the entire state space come in. The collective term for these various algorithms is \emph{Reinforcement Learning}.

Since the transition probabilities and immediate rewards are not known, these algorithms first need to create samples from which to estimate them. These samples are created using a behavior. So this behavior has to explore the state-action space. And from the random variables generated by this behavior, the algorithm has to estimate the value functions. 

If the state space is small and we only have the first problem, we can separate the exploration from the later exploitation of our knowledge about the MDP. These methods are grouped under the term \emph{Batch Reinforcement Learning}. But if our state space is too large for that, it is impractical to wait for this. So the behavior needs to get updated with the information gathered continuously; the algorithm needs to be what is called \emph{online}. But note that since Batch Reinforcement Learning algorithms often utilized little information more efficiently than their online counterparts, there were some efforts to modify these algorithms to allow for continuous updates of the estimates of the value functions. These were categorized as \emph{growing batch reinforcement learning} algorithms in contrast to pure batch learning \parencite{langeBatchReinforcementLearning2012} blurring the lines in the process.

So one of the main problems in online learning is balancing the exploration with the exploitation of the knowledge gathered. 

%\section{Asynchronous Dynamic Programming}
\section{Monte-Carlo}
\section{Temporal Difference Learning TD}
\section{Mixing Both -- The Generalization TD(\(\gamma\))}

\endinput


%%
%% Falls Abbildungen und/oder Tabellen vorhanden,
%% entsprechend auskommentieren 
%%
%\clearpage
%\listoffigures
%\listoftables
%%
%% Anhang (falls nicht vorhanden, auskommentieren)
%%
\begin{appendix}
% \chapter{Title of the appendix}

\endinput

\end{appendix}
\backmatter
%%
%% Literaturverzeichnis (bitte einkommentieren!)
%%
\printbibliography

%\begin{thebibliography}{99}
%%% Beispiele:
%%
%% Buecher, Dissertationen etc.:
%%
%% \bibitem{Ba02}
%% H.~Bauer, \emph{Wahrscheinlichkeitstheorie}, 4.~Auflage, de~Gruyter, Berlin,
%%   New York, 2002.
%%
%%
%% Artikel in Journalen:
%%
%% \bibitem{Fe52}
%% W.~Feller, The parabolic differential equations and the associated
%%   semi-groups of transformations, \emph{Ann. Math.} \textbf{3} (1952), 468--519.
%%
%%
%% Artikel in Proceedings, Sammelbaenden etc.:
%%
%% \bibitem{BaPi89}
%% M.~Barlow, J.~Pitman, und M.~Yor, On Walsh's Brownian motion,
%%   \emph{S{\'e}minaire de Probabilit{\'e}s XXIII} (J.~Az{\`e}ma, P.~A. Meyer, und
%%   M.~Yor, Hrsg.), Lecture Notes in Mathematics, no. 1372, Springer--Verlag,
%%   Berlin, Heidelberg, New York, 1989, pp.~275--293.
%%
%%
%% ACHTUNG: es gibt SEHR viele andere Stile fuer die Zitation als die obigen!
%% => Betreuer fragen!
%%

\endinput

%\end{thebibliography}
\end{document}

