\chapter{Markov Decision Processes}
\begin{definition}(Kernel)
	\((Y,\cA_Y), (X,\cA_X)\) measure spaces\\
	 \(\lambda\colon X\times\cA_Y\to \R\) is a \emph{(probability) kernel} \(
	 :\iff \begin{aligned}[t]
	 &\lambda(\cdot,A)\colon x\mapsto \lambda(x,A) \text{ measurable}\\
	 &\lambda(x,\cdot)\colon A\mapsto \lambda(x,A) \text{ a (prob.) measure}
	 \end{aligned}
	  \)\\
	  Since we will interpret probability kernels as distributions over \(Y\) given a  certain condition \(X\), the notation \(\lambda(\cdot\mid x) \coloneqq \lambda(x,\cdot)\) helps this intuition. 
\end{definition}
\begin{definition}(Markov Decision Process - MDP)\\
\(\cM=(\cX,\cA,\cP_0) \), with:
	
\begin{tabular}{l l l}
	\(\cX\) & countable (finite) set of states&\\
	\(\cA\) & countable (finite) set of actions &\\
	\multicolumn{2}{l}{
		\(
		\begin{cases}
		\cX\times \cA \to \pmeas(\cX\times\R)\\
		(x,a)\mapsto \cP_0(\cdot \mid x,a)
		\end{cases}
		\)
	}  &
	\parbox[h]{17em}{
		\emph{transition probability kernel} \\
		\(\pmeas(\cX\times\R) \) the set of probability measures on \(\cX\times\R \), \\
		\(\cX\) represents the next states,\\
		\(\R\) the payoffs
	}
	\end{tabular}\\
is a \emph{(finite) Markov Decision Process}.\\
Together with a discount factor \(\gamma\in[0,1]\) it is a:\\
\begin{tabular}{l l}
	\emph{discounted reward} MDP & \(\gamma <1 \)\\
	\emph{undiscounted reward} MDP & \(\gamma=1 \)
\end{tabular}\\
For \((Y_{(x,a)}, R_{(x,a)})\sim \cP_0(\cdot\mid x,a) \) a random variable, is
\[r(x,a)\coloneqq \E[R_{(x,a)}] \quad\text{ the \emph{immediate reward function}}\]
An MDP is \emph{evaluated} as follows:\\
1. Select the initial state \(X_0\) an \(\cX\)-valued random variable.\\ 
2. \((A_t, t\in \N)\) action selection rules (behaviors) will be discussed later, for now simply assume \(\cA\)-valued random variables.\\
3. Select inductively: \((X_{t+1}, R_{t+1})\sim\cP_0(\cdot\mid X_t, A_t)\) with the markov property, i.e.:
\begin{align*} 
&\Pr[(X_{t+1},R_{t+1})=(x,r) \mid (X_t,A_t)=(x_t,a_t),\dots, (X_0,A_0)=(x_0,a_0)]\\
&=\Pr[(X_{t+1}, R_{t+1})=(x,r)\mid (X_t,A_t)=(x_t,a_t)]
\end{align*}
resulting in the stochastic process \(((X_t,A_t,R_{t+1}), t\geq 0)\), which allows to define the \emph{return}:
\[\cR\coloneqq\sum_{t=0}^{\infty}\gamma^t R_{t+1}\]
\end{definition}
%(

\begin{remark}
\((X_{t+1}, R_{t+1})\sim\cP_0(\cdot\mid X_t, A_t)\) with the markov property is well defined, i.e.:
\begin{align*}
	&\exists (X_{t+1}, R_{t+1})\ \cX\times\R\text{-valued random variable}: \\ 
	&(X_{t+1}, R_{t+1})\sim\cP_0(\cdot\mid X_t, A_t) \text{ and satisfies the markov property}
\end{align*}
\end{remark}
\begin{proof}
	%TODO
\end{proof}
\begin{remark}\leavevmode
\begin{enumerate}
	\item From now on we assume that \(\forall (x,a)\in\cX\times\cA:|R_{(x,a)}|\le R\in\R\) almost surely. This also implies: 			\(
	\begin{aligned}[t]
		&\|r\|_{\infty}=\sup_{(x,a)\in\cX\times\cA}|\E[R_{(x,a)}]|\le R\\
		&|\cR|\le\sum_{t=0}^\infty \gamma^t |R_{t+1}|\le \frac{R}{1-\gamma} \text{ a.s.}
	\end{aligned}
	\)
	\item Sometimes not all actions make sense in all states. A simple fix would be to set the immediate reward functions for those actions very low, or (if possible) redirect them to the closest possible action. \\
	A more formal approach would be to introduce an additional mapping, which assigns the set of admissible actions to each state \(\cX\to\cP(\cA)\), or alternatively define a (binary) relation on \(\cX\times\cA\).
	\item If there is just one admissible action in every state, the MDP is equivalent to a normal Markov Process.
	\item Instead of a transition probability kernel \(\cP_0\), sometimes a \emph{transition function} f with a and an exogenous random element \(D_t\) (e.g. Demand) is used to define the next state and reward: \((X_{t+1},R_{t+1})=f(X_t,A_t,D_t)\) 
\end{enumerate}
\end{remark}
\begin{definition}\(\cM=(\cX,\cA,\cP_0)\) a MDP\\
	\(x\in\cX\) is a \emph{terminal (absorbing)} state \(:\iff \forall s\in\N: \Pr(X_{t+s}=x\mid X_t=x)=1\)\\
	An MDP with such states is called \emph{episodic}.\\
	An \emph{episode} is the random time period \((1,\dots,T)\) until a terminal state is reached.
\end{definition}
\begin{remark}\leavevmode
	\begin{itemize}
		\item The reward in a terminal state is by convention zero, i.e. \(x\) terminal state implies \(\forall a\in\cA: R_{(x,a)}=0\).
		\item Episodic MDPs are often undiscounted
	\end{itemize}	
\end{remark}
\begin{definition}\(\cM=(\cX,\cA,\cP_0)\) a MDP\\
	 An \(A_t\) selection-rule \(\pi=(\pi_t,t\in\N_0)\) is called \emph{behavior}, where
	 \[ 
	 	\pi_t\colon
	 	\begin{cases}
	 		((\cX\times\cA\times\R)^t\times\cX)\times\cP(\cA) \to \R \\
	 		(y,A)\mapsto \pi_t(A\mid y)
	 	\end{cases} \text{ is a probability kernel}
	 \]
	 and \(A_t\sim \pi_t(\cdot\mid (X_0,A_0,R_1), \dots,(X_{t-1},A_{t-1},R_t),X_t))\)\\
	 Special cases:
	 \begin{enumerate}
	 	\item \emph{Determinisitic stationary policies} specified with some abuse of notation:
	 	\[\leavevmode \pi\colon\cX\to\cA \text{ with }A_t=\pi(X_t)\]
	 	\item \emph{(Stochastic) stationary policies} specified by:
	 	\[\pi\colon \begin{cases}
	 	\cX\times\cP(\cA)\to\R\\
	 	(x,A)\mapsto \pi(A\mid x)
	 	\end{cases} \text{ with } A_t\sim\pi(\cdot\mid x)
	 	\]
	 \end{enumerate}
	 \(\statPolicy\) is the \emph{set of (stoch.) stationary policies}, \\ 
	 \(\detPolicy\) is the \emph{set of deterministic stationary policies} (note \(\detPolicy\subseteq\statPolicy \))
\end{definition}
\begin{remark}
A stationary policy induces a \emph{time-homogenous} markov chain.
\end{remark}
\begin{definition}(Markov Reward Process - MRP)
%TODO
\end{definition}
\section{Value functions}
The goal in this section is to
\begin{itemize}[itemsep=0pt, topsep=1pt]
\item define Value functions which assign states (and actions) a value, which allow the agent to make a more nuanced decisions than comparing immediate rewards of different actions
\item explore the relation of different value functions
\item show uniqueness of optimal value functions with the Banach fixpoint theorem, yielding a simple approximation methode along the way
\item demonstrate that in MDPs deterministic stationary policies are generally a large enough set of policies to choose from
 \end{itemize}
\begin{definition}\(\cM=(\cX,\cA,\cP_0)\) MDP, \(\pi\) Behavior\\
Select \(X_0\) such that \(\forall x\in\cX:\Pr(X_0=x)>0\) and evaluate the MDP with
\(((X_t,A_t,R_{t+1}), t\in \N_0)\) the resulting stoch. process.\\
\def\arraystretch{3}
\[
\begin{array}{l l}
	V^\pi\colon
	\begin{cases}
		\cX\to\R\\
		x\mapsto \E[\cR\mid X_0=x]
	\end{cases} 
	& \text{is the \emph{value function} for } \pi \footnotemark\\
	Q^\pi\colon
	\begin{cases}
		\cX\times\cA\to\R\\
		(x,a)\mapsto \E[\cR\mid X_0=x, A_0=a]
	\end{cases}
	& \text{is the \emph{action value function} for } \pi \footnotemark\\
	V^*\colon
	\begin{cases}
		\cX\to\R\\
		x\mapsto \sup\limits_{\pi\text{ Behav.}} V^\pi(x)
	\end{cases} 
	& \text{is the \emph{optimal value function}}\\
	Q^*\colon
	\begin{cases}
		\cX\times\cA\to\R\\
		(x,a)\mapsto \sup\limits_{\pi\text{ Behav.}} Q^\pi(x,a)
	\end{cases}
	& \text{is the \emph{optimal action value function}}
\end{array}
\]
\(\pi\) is \emph{optimal} \(:\iff V^*=V^\pi\)
\end{definition}
\footnotetext[1]{Well defined because \(\Pr(X_0=x)>0\)}
\footnotetext{Well defined because \(A_1\sim \pi_1(\cdot\mid (x,a,r_0), x_1)\) is defined for all \(a\) regardless of \(\pi_0\)}
\begin{definition}\(\cM=(\cX,\cA,\cP_0)\) MDP\\
Sometimes we don't care about the probability distribution of the reward, so we define:
	\[
	p\colon 
	\begin{cases}
		\cX\times\cA\times\cP(\cX)\to\R\\
		(x,a,Y)\mapsto \cP_0(Y\times\R\mid x,a)
	\end{cases}\text{ the \emph{\underline{state} transition kernel}.}
	\] 
And use the notation \(p(y\mid x,a)\coloneqq p(\{y\}\mid x,a)\) with \((x,a,y)\in\cX\times\cA\times\cX\)
\end{definition}
\begin{prop}\(\cM=(\cX,\cA,\cP_0)\) MDP, \(\pi\in\detPolicy\) 
	\[Q^\pi(x,a)=r(x,a)+\gamma\sum_{y\in\cX}p(y\mid x,a)V^\pi(y)	\]
\end{prop}
\begin{proof}
\end{proof}
\begin{corollary}\label{V^pi,Q^pi relation}  \(\cM=(\cX,\cA,\cP_0)\) MDP, \(\pi\in\detPolicy\) 
\begin{itemize}
	\item \(V^\pi(x)=Q^\pi(x,\pi(x))\)
	\item \(V^\pi=r(x,\pi(x))+\gamma\sum\limits_{y\in\cX}p(y\mid x,\pi(x))V^\pi(y) \)
\end{itemize}
\end{corollary}
\begin{definition}\(\cM=(\cX,\cA,\cP_0)\) MDP, \(\pi\in\detPolicy\) \\
The mapping \(T^\pi\colon\R^\cX\to\R^\cX\) with:
	\[
	T^\pi V(x)\coloneqq r(x,\pi(x))+\gamma\sum_{y\in\cX}p(y\mid x,\pi(x)) V(y)\qquad V\in\R^\cX, x\in\cX
	\]
is called the \emph{Bellman Operator}
\end{definition}
\begin{remark}\leavevmode
	\begin{enumerate}
	\item From \ref{V^pi,Q^pi relation} follows \(\forall\pi\in\detPolicy : T^\pi V^\pi=V^\pi\)
	\item \(T^\pi\) meets the requirements of the Banach fixed-point theorem for \({\gamma<1}\), this implies that \(V^\pi\) for \(\pi\in\detPolicy\)
	is a \emph{unique} fixpoint and can be approximated with the canonical iteration
	\item \(T^\pi\) is an affine operator
	\item \(W_1,W_2\in\R^\cX\) then 
	\[\forall x\in\cX:W_1(x)\le W_2(x) \implies \forall x\in\cX:T^\pi W_1(x)\le T^\pi W_2(x)\]
	\end{enumerate}
\end{remark}
\begin{proof}
\end{proof}
\begin{definition}\(\cM=(\cX,\cA,\cP_0)\) MDP
\begin{align*}
	\tilde{V}(x)&\coloneqq \sup_{\pi\in\detPolicy} V^\pi(x)\\
	\tilde{Q}(x,a)&\coloneqq\sup_{\pi\in\detPolicy}Q^\pi(x,a)
\end{align*}
\end{definition}
\begin{definition}\(\cM=(\cX,\cA,\cP_0)\) MDP\\
The mapping \(T^*\colon\R^\cX\to\R^\cX\) with:
	\[
	 T^* V(x)\coloneqq \sup_{a\in\cA}\left\{r(x,a)+\sum_{y\in\cX}p(y\mid x,a) V(y)\right\} \qquad V\in\R^\cX, x\in\cX
	\]
is called the \emph{Bellman Optimality Operator}
\end{definition}
\begin{lemma}\label{V*,Q* relation}\(\cM=(\cX,\cA,\cP_0)\) MDP
\begin{enumerate}[label=\textbf{(\roman*)},font=\normalfont]
\item \(\tilde{V}(x)\coloneqq\sup\limits_{a\in\cA} \tilde{Q}(x,a)\)
\item \( \tilde{Q}(x,a) = r(x,a)+\gamma\sum\limits_{y\in\cX}p(y\mid x,a)\tilde{V}(y) \)
\item\label{i:3} \(V^*(x)=\sup\limits_{a\in\cA}Q^*(x,a)\)
\item\label{i:4} \(Q^*(x,a)=r(x,a)+\gamma \sum\limits_{y\in\cX}p(y\mid x,a) V^*(y)\)
\end{enumerate}
\end{lemma}
\begin{proof}
	%TODO
\end{proof}

\begin{corollary}
	\begin{align*}
	&T^*\tilde{V}=\tilde{V}\\
	&T^*V^*=V^*
	\end{align*}
\end{corollary}
\begin{proof}
\begin{align*}
	V^*(x)\xeq{\ref{i:3}}\sup_{a\in\cA}Q^*(x,a)\xeq{\ref{i:4}}\sup_{a\in\cA}\left\{r(x,a)+\sum_{y\in\cX}p(y\mid x,a) V^*(y)\right\} =T^*V^*(x)
\end{align*}
\(\tilde{V}\) analogous
\end{proof}
\begin{thm}\(\cM=(\cX,\cA,\cP_0)\) MDP\\
\(T^*\) satisfies the requirements of the Banach fixpoint theorem, in particular:
	\[V^*(x)=\sup_{\pi\in\statPolicy}V^\pi(x)=\tilde{V}(x) \]
is the unique fixpoint of \(T^*\)
\end{thm}
\begin{lemma}(Blackwell's condition for contraction)
\end{lemma}
\begin{proof}
https://math.stackexchange.com/questions/1087885/blackwells-condition-for-a-contraction-why-is-boundedness-neccessary?rq=1
\end{proof}
\begin{proof}[Proof (Theorem)]
\end{proof}
\begin{prop}\label{sup is attained}\(\cM=(\cX,\cA,\cP_0)\) MDP\\
The following statements are equivalent:
\begin{enumerate}[label={(\roman*)},font=\normalfont]
\item \(\pi \in\statPolicy\) is optimal (\(V^*=V^\pi\))
\item \(\forall x\in\cX: V^*(x)=\sum\limits_{a\in\cA}\pi(a\mid x)Q^*(x,a)\)
\item\label{ii:3} \(\forall x\in\cX: \pi=\arg\max\limits_{\pi\in\statPolicy}\sum\limits_{a\in\cA}\pi(a\mid x)Q^*(x,a) \)
\item \(\pi(a\mid x)>0 \iff Q^*(x,a)=V^*(x) =\sup\limits_{b\in\cA}Q*(x,b)\) \\
	``actions are concentrated on the set of actions that maximize \(Q^*(x,\cdot)\)''\\
	(this also implies: \(Q^*(x,a)<V^*(x) \implies \pi(a\mid x)=0\))
\end{enumerate}
\end{prop}
\begin{proof}
%TODO
\end{proof}
\begin{definition}
	\(Q\colon\cX\times\cA\to\R\) an action value function, \(\tilde{\pi}\colon\cX\to\cA\) with:
	\[
	\tilde{\pi}(x)\coloneqq\arg\max_{\pi\in\statPolicy}\sum_{a\in\cA}\pi(a\mid x) Q(x,a)\qquad x\in\cX
	\]
	\(\tilde{\pi}(x)\) is called \emph{greedy} with respect to Q in \(x\in\cX\)\\
	\(\tilde{\pi}\) is called \emph{greedy} w.r.t. Q
\end{definition}

\begin{remark}\leavevmode
	\begin{itemize}
	\item \ref{sup is attained}\ref{ii:3} implies that greedy w.r.t. \(Q^*\) is optimal. 
	This means that knowledge of \(Q^*\) is sufficient to select the best action.
	\item \ref{V*,Q* relation} implies that knowledge of \(V^*,r,p\) is sufficient as well.
	\end{itemize}
\end{remark}
\endinput
