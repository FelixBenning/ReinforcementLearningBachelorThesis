\chapter{Appendix}
\section{Basic Probability Theory}
\begin{lemma}\leavevmode \label{appx1}
    \begin{enumerate}[label=(\roman*), font=\normalfont]
        \item\label{appx1:i} \(\Pr(A\cap B\mid C) = \Pr(A \mid B \cap C)\Pr(B\mid C) \)
        \item\label{appx1:ii} \(\Pr(A\mid C)=\sum\limits_{n\in\N} \Pr(A \mid B_n \cap C)\Pr(B_n\mid C)  \)
        for  \(\Pr\left(\biguplus_{n\in\N}B_n\right)=1 \)
        \item\label{appx1:iii} \(\E[X\mid C]=\sum\limits_{n\in\N} \E[X\mid C \cap B_n] \Pr(B_n \mid C) \) for  \(\Pr\left(\biguplus_{n\in\N}B_n\right)=1 \)
    \end{enumerate}
\end{lemma}
\begin{proof}
    \ref{appx1:i}
    \[
        \Pr(A\cap B\mid C)
        =\frac{\Pr(A\cap B\cap C)}{\Pr(B\cap C)}\frac{\Pr(B\cap C)}{\Pr(C)}
        =\Pr(A\mid B\cap C)\Pr(B\mid C)
    \]
    \ref{appx1:ii}
    \begin{align*}
        \Pr(A\mid C)&=\Pr\left(A\cap \biguplus_{n\in\N} B_n \;\middle|\; C\right)
        =\sum_{n\in\N} \Pr(A\cap B_n\mid C)\\
        &\lxeq{\text{\ref{appx1:i}}} \sum_{n\in\N} \Pr(A \mid B_n \cap C)\Pr(B_n\mid C)
    \end{align*}
    \ref{appx1:iii}
    \begin{align*}
        \E[X\mid C]
        &=\frac{1}{\Pr(C)} \int_C X d\Pr 
        =\sum_{n\in\N}\frac{1}{\Pr(C)} \int_{C\cap B_n} X d\Pr\\
        &=\sum_{n\in\N} \frac{\Pr(C\cap B_n)}{\Pr(C)} \frac{1}{\Pr(C\cap B_n)} 
        \int_{C\cap B_n} X d\Pr \\
        &=\sum_{n\in\N}\E[X\mid C\cap B_n]\Pr(B_n\mid C) \qedhere
    \end{align*}
\end{proof}
\begin{lemma}\label{appx2}
    Let \((\Omega,\cA,\mu)\) be a measure space and a function \(f\) exists with
    \begin{align*}
        &f\colon \Omega \to \R \text{ injective and measureable,} \\
        &f^{-1}\colon f(\Omega)\to \Omega \text{ measureable.}
    \end{align*}
    Then for \(X\) \(\Omega\)-valued random variable and
    \(Y\) \(f(\Omega)\)-valued random variable
    \[
        \Pr_{f\circ X} =\Pr_Y \iff \Pr_X=\Pr_{f^{-1}\circ Y}
    \]
\end{lemma}
\begin{proof}
    ``\(\Leftarrow\)'' Let \(A\in\cB(\R)\), then w.l.o.g. \(A\subseteq f(\Omega)\) otherwise
    \[
        \Pr_{f\circ X}(A)=\Pr(A\cap f(\Omega))+\underbracket{
            \Pr_{f\circ X}(A\cap f(\Omega)^\complement)
            }_{=0}=\dotsc=\Pr_Y(A) 
    \]
    Thus \(f\circ f^{-1}(A)=A\) holds, which finishes this direction with
    \begin{align*}
        \Pr_{f\circ X}(A)
        &=\Pr(X^{-1}\circ f^{-1}(A))=\Pr_X(f^{-1}(A))\\
        &=Pr_{f^{-1}\circ Y}(f^{-1}(A))=\Pr(Y^{-1}\circ f\circ f^{-1}(A))\\
        &=\Pr_Y(A)
    \end{align*}
    ``\(\Rightarrow\)'' Let \(A\in\cA\), then
    \begin{align*}
        \Pr_X(A)&=\Pr(X^{-1}\circ f^{-1}\circ f(A))=\Pr_{f\circ X}(f(A))\\
        &=\Pr_Y(f(A))=\Pr(Y^{-1}\circ f(A))\\
        &=\Pr_{f^{-1}\circ Y}(A) \qedhere
    \end{align*}
\end{proof}
\begin{definition}[Pseudo-inverse]\label{appx3}
    Let \(F\) be a cumulative distribution function, then 
    \[F^\leftarrow (y) \coloneqq \inf\{x\in\R : F(x)\ge y \}\]
    is called the \emph{Pseudo-inverse} of \(F\).
\end{definition}
\begin{lemma}
    Let \(F\) be a cdf, then
    \begin{enumerate}[label=(\roman*), font=\normalfont]
        \item\label{appx3:i} \(
            F^\leftarrow (y)\le x \iff y \le F(x)
        \)
        \item\label{appx3:ii} \(U\sim\cU(0,1) \implies F^\leftarrow(U) \sim F \)
    \end{enumerate}
\end{lemma}
\begin{proof}
    \ref{appx3:i} ``\(\Rightarrow\)''
    \begin{align*}
        y&\le \inf_{x\in \{z\in\R: F(z)\ge y\}} F(x) 
        \xeq{\substack{
            \text{F right-}\\
            \text{continuous}}
        } F(\inf\{z\in\R: F(z)\ge y\})
        \xeq{\text{def.}} F(F^\leftarrow(y))\\
        &\le F(x)
    \end{align*}
    Where the last inequality follows from the assumption \(F^\leftarrow (y)\le x\) and F being non decreasing.

    \noindent ``\(\Leftarrow\)'' Follows simply from the fact that x is included in the set of the infimum. 
    \begin{align*}
        y\le F(x) \implies F^\leftarrow (y) = \inf\{z\in\R:F(z)\ge y\} \le x
    \end{align*}

    \noindent \ref{appx3:ii} is a simple corollary from \ref{appx3:i}
    \[
        \Pr(F^\leftarrow (U)\le x) 
        \xeq{\text{\ref{appx3:i}}} \Pr(U\le F(x)) = F(x) 
        \qedhere
    \]
\end{proof}
\section{Analysis}
\endinput
